\subsubsection{3rd. Planetocentric stage }
\paragraph{} Once the spacecraft, enter into the SOI of the destination planet, the planetocentric stage starts. The spacecraft, describes a planet arrival hyperbola. Depending of the arrival velocity and position three main scenarios can be described for the spacecraft.

\begin{description}
	\item[Impact] If arrival position $r_p \leq R_P$.
	\item[Slowdown] Atmosphere interaction will slowdown the spacecraft if $r_p \leq R_a$.
	\item[Flyby] Spacecraft will perform a flyby over the planet if $r_p > R_a$ or $b > b_a$.	
\end{description}

Note that $R_P$ and $R_a$, correspond to the planet and atmosphere radius respectively.

The spacecraft, will only stay trapped on the destination planet, orbiting, if the spaceship has slowdown enough in contact with the planet atmosphere or the spaceship braked during the periastron.

Because, the main assignment objective was to calculate, a trajectory between to planets for interplanetary travel. This stage is been only develop theoretically, expressions for computing theses parameters can be obtained on, \cite{PCA}. First and third stage of the PCA, share some concepts and equations because in both, an hyperbolic trajectory is performed.