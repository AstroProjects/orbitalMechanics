\chapter{Conclusions}
\paragraph{} After performing this project, it can be said that orbital mechanics is not a trivial matter. This project, have been hard to start, because it took us some time to be familiar with the concepts and how they are explained through the equations. Finally, after this project, we understand the patched conics approximation development and hypothesis.

It has not been easy to implement the method explained during the lessons, mostly because at some point if concepts weren't clear enough, the convergence sections of the code, failed to give a solution. Getting the code working well to verify the examples, excited us, we feel confident to have consistent results. After this personal thoughts introduction we can extract some technical conclusions.

Main code implemented, demonstrated to work optimally for trajectories departing Earth to outer planets. Both trajectories, elliptical and hyperbolic, have been correctly compute as seen on the code verification. Nevertheless, as more cases were tried more specific problems for each case appeared. This fact, lead us to improve the code at every simulation.

Comparing some results from book \cite{llibreVictor} examples with our code, in most cases parameters match. The code, only presents troubles calculating inner planets trajectories, most likely, because, proper exceptions for this trajectories have not been implemented yet. 

One final thought, after computing these trajectories, it is seen that years for travelling to Mars are coming. Around 2022, trips to Mars from Earth will took really few days, around 100 days. This ideas totally matches with the fact that NASA and SpaceX, are developing their deep space exploration rockets for stepping humans into the red planet.

Finally, patched conics approximation, demonstrated that despite its complexity, it is a low computational effort method. These results will be taken as starting parameters for precise numerical integrations on real missions.