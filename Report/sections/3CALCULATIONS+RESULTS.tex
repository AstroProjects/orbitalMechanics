\chapter{Calculations and results}

\section{Verification}
In order to verify the code, examples shown at reference \cite{CalafEnunciat} will be used. In this section, the results provided for every example will be exposed together with the results obtained using the code developed for this project and conclusions will be extracted.
\subsection{From Earth to Mars using an elliptic heliocentric trajectory}
\begin{table}[H]
\centering
\begin{tabular}{|lc|}
\hline
Departure date              & 2020 Jul 19                \\ 
Arrival date                & 2021 Gen 25                \\ 
$\Delta$t                    & 190 days                   \\ 
$r_1$                          & (0.4537, -0.9094, 0.0000)AU  \\ 
$r_2$                          & (0.3148, 1.5078, 0.0239)AU   \\ \hline
\end{tabular}
\caption{Data provided by the first example}
\end{table}

\begin{table}[H]
\centering
\begin{tabular}{|lc|}
\hline
Semimajor axis                          & 1.33069 AU      \\ 
Eccentricity                           & 0.23629         \\ 
$\theta _0$                     & 359.621\degree                 \\ 
$\omega$                           & 0.470\degree                                 \\ 
Inclination                          & 1.435\degree                             \\ 
$\Omega$                & 296.424º                   \\ 
Heliocentric velocity at departure (km/s) & (29.3678, 14.6982, 0.8229) \\ 
Heliocentric velocity at arrival (km/s) & (20.4069,8.2271,0.3656)    \\
\hline
\end{tabular}
\caption{Results provided by the first example}
\end{table}

\begin{table}[H]
\centering
\begin{tabular}{|lc|}
\hline
Semimajor axis                          & 1.33073 AU      \\ 
Eccentricity                           & 0.236291         \\ 
$\theta _0$         & 359.613\degree                   \\
$\omega$                           & 0.386861\degree                            \\ 
Inclination                          & 1.43388\degree                             \\ 
$\Omega$                & 296.515\degree                                   \\ 
Heliocentric velocity at departure (km/s) & (29.367,14.6986,0.822024) \\ 
Heliocentric velocity at arrival (km/s)&   (-20.4068,8.27743,-0.364583) \\
\hline
\end{tabular}
\caption{Results computed by the code developed for the first example}
\end{table}


\begin{table}[H]
\centering
\begin{tabular}{|cc|}
\hline
\textbf{Parameter}   & \textbf{Relative error (\%)} \\ \hline
a           & 0.0030              \\
e           & 0.0004              \\
$\theta _0$ & 0.0022              \\
$\omega$    & 17.6891              \\
i           & 0.0780              \\
$\Omega$    & 0.0307              \\
$v_{t_1}$   & 0.0017              \\
$v_{t_1}$   & 0.0001              \\ \hline
\end{tabular}
\caption{Relative error in the first example}
\end{table}

\subsection{From Mars to Jupiter using an elliptic heliocentric trajectory}
\begin{table}[H]
\centering
\begin{tabular}{|lc|}
\hline
Departure date              & 2026 Jun 05                \\ 
Arrival date                & 2029 April 25                \\ 
$\Delta$t                    & 1055 days                   \\ 
$r_1$                          & (1.3277, 0.4901, 0.0223)AU  \\ 
$r_2$                          & (5.0135, 2.1380, 0.0505)AU   \\ \hline
\end{tabular}
\caption{Data provided by the second example}
\end{table}


\begin{table}[H]
\centering
\begin{tabular}{|lc|}
\hline
Semimajor axis                          & 3.45403 AU      \\ 
Eccentricity                           & 0.59218         \\ 
$\theta _0$                     & 350.769\degree                 \\ 
$\omega$                           & 182.312\degree                                 \\ 
Inclination                          & 7.513\degree                             \\ 
$\Omega$                & 207.121º                   \\ 
Heliocentric velocity at departure (km/s) & (12.5324, 28.6817, 4.1200) \\ 
Heliocentric velocity at arrival (km/s) & (1.9715,7.9799,1.0552)    \\
\hline
\end{tabular}
\caption{Results provided by the second example}
\end{table}

\begin{table}[H]
\centering
\begin{tabular}{|lc|}
\hline
Semimajor axis    &  3.45405    \\ 
Eccentricity       & 0.592181        \\ 
$\theta _0$                &     350.469\degree \\
$\omega$             & 196.156\degree                            \\ 
Inclination                          & 7.508444\degree                             \\ 
$\Omega$          & 207.127\degree                                   \\ 
Heliocentric velocity at departure (km/s) & (-19.0894,24.8148,-4.05809)\\ 
Heliocentric velocity at arrival (km/s)&    (3.83725,-7.26513,1.08284)\\
\hline
\end{tabular}
\caption{Results computed by the code developed for the second example}
\end{table}

\begin{table}[H]
\centering
\begin{tabular}{|cc|}
\hline
\textbf{Parameter}   & \textbf{Relative error (\%)} \\ \hline
a           & 0.0006              \\
e           & 0.0002              \\
$\theta _0$ & 0.0000              \\
$\omega$    & 7.5936              \\
i           & 0.0606              \\
$\Omega$    & 0.0029              \\
$v_{t_1}$   & 0.0014              \\
$v_{t_1}$   & 0.0001              \\ \hline
\end{tabular}
\caption{Relative error in the second example}
\end{table}

\subsection{From Earth to Mars using an hyperbolic heliocentric trajectory}

\begin{table}[H]
\centering
\begin{tabular}{|lc|}
\hline
Departure date              & 2020 Mar 06                \\ 
Arrival date                & 2020 Jun 09 \\ 
$\Delta$t                    & 95 days                   \\ 
$r_1$                          & (-0.9609, 0.2466, 0.0000)AU  \\ 
$r_2$                          & (0.7285, -1.1980,-0.0430)AU   \\ \hline
\end{tabular}
\caption{Data provided by the third example}
\end{table}

\begin{table}[H]
\centering
\begin{tabular}{|lc|}
\hline
Semimajor axis         & 71.08581 AU      \\ 
Eccentricity                           & 1.01113         \\ 
$\theta _0$                     & -53.310\degree                 \\ 
$\omega$                           & 233.297\degree                                 \\ 
Inclination                          & 2.513\degree                             \\ 
$\Omega$                & 345.619º                   \\ 
Heliocentric velocity at departure (km/s) & (9.1364, -41.4090, -1.6612) \\ 
Heliocentric velocity at arrival (km/s) & (35.1754,-6.3201,0.1148)    \\
\hline
\end{tabular}
\caption{Results provided by the third example}
\end{table}

\begin{table}[H]
\centering
\begin{tabular}{|lc|}
\hline
Semimajor axis       &   71.6165   \\ 
Eccentricity              & 1.01105       \\ 
$\theta _0$      &   306.691\degree      \\
$\omega$            & 233.309\degree                            \\ 
Inclination                & 2.51416\degree                             \\ 
$\Omega$            & 345.607\degree                                   \\ 
Heliocentric velocity at departure (km/s) & (9.13562,-41.4082,-1.66139)\\ 
Heliocentric velocity at arrival (km/s)&    (35.1747,-6.31841,0.115199)\\
\hline
\end{tabular}
\caption{Results computed by the code developed for the third example}
\end{table}

\begin{table}[H]
\centering
\begin{tabular}{|cc|}
\hline
\textbf{Parameter}   & \textbf{Relative error (\%)} \\ \hline
a           & 0.7465              \\
e           & 0.0079              \\
$\theta _0$ & 0.0003              \\
$\omega$    & 0.0051              \\
i           & 0.0462              \\
$\Omega$    & 0.0035              \\
$v_{t_1}$   & 0.1189              \\
$v_{t_1}$   & 0.0028              \\ \hline
\end{tabular}
\caption{Relative error in the third example}
\end{table}
\subsection{Verification conclusions}
It can be appreciated that in most of the cases the relative error is far less than one. The highest error produced is less than 17\%, and its cause probably deal with the equations and assumptions done while computing the results.

Although, heliocentric arrival and departure velocities, may differ in some sign for one of the components. After an extensive study, is determined that could be, because of the angles definition. This opposite values, show that the vector obtained is just the symmetric, that is equivalent to the solution if angles are chosen accordingly

Therefore, the code is taken as valid and can be used in order to obtain other interplanetary trajectories. It has been designed in a general manner so the same code can compute elliptic and hyperbolic trajectories with any departure and arrival planet in the solar system. 

\section{Main interplanetary orbit calculations}
Following, the cases proposed in reference \cite{CalafEnunciat} will be solved. 
\subsection{Case 1: Mars to Jupiter}

\begin{table}[H]
\centering
\begin{tabular}{|lc|}
\hline
Departure date              & 2037 Oct 25                \\ 
Arrival date                & 2039 Oct 15 \\ 
$\Delta$t                    & 720 days                   \\ 
$r_1$                          & (1.0707, 0.9868, 0.0055)  \\ 
$r_2$                          & (5.2210, 1.4357,0.1109)   \\ \hline
\end{tabular}
\caption{Data provided for a travel between Mars and Jupiter}
\end{table}

\begin{table}[H]
\centering
\begin{tabular}{|lc|}
\hline
Semimajor axis       &  4.27012 AU  \\ 
Eccentricity         &    0.725509   \\ 
$\theta _0$      &  302.422 \degree      \\
$\omega$            & 63.3568\degree                            \\ 
Inclination        & 2.14985\degree                             \\ 
$\Omega$        & 36.89\degree                                   \\ 
Heliocentric velocity at departure (km/s) & (-29.1352,12.6808,1.03727) \\ 
Heliocentric velocity at arrival (km/s)&    (4.72014,-9.86931,-0.402679)\\
\hline
\end{tabular}
\caption{Results computed for a travel between Mars and Jupiter}
\end{table}

\subsection{Case 2: Earth to Mars}

\begin{table}[H]
\centering
\begin{tabular}{|lc|}
\hline
Departure date              & 2033 Mar 13                \\ 
Arrival date                & 2033 Aug 05 \\ 
$\Delta$t                    & 145 days                   \\ 
$r_1$                          & (0.9848, 0.1338, 0.0000)  \\ 
$r_2$                          & (0.6797, 1.2298,0.0424)   \\ \hline
\end{tabular}
\caption{Data provided for case 2}
\end{table} 

\begin{table}[H]
\centering
\begin{tabular}{|lc|}
\hline
Semimajor axis     & 1.37053 AU \\ 
Eccentricity        &    0.615632   \\ 
$\theta _0$    &   256.507\degree      \\
$\omega$        & 103.493\degree                            \\ 
Inclination         & 2.15438\degree                             \\ 
$\Omega$            & 7.73712\degree                                   \\ 
Heliocentric velocity at departure (km/s) & (-22.8707,24.7746,1.03934)\\ 
Heliocentric velocity at arrival (km/s)&    (9.73264,-22.7879,-0.898741)\\
\hline
\end{tabular}
\caption{Results computed for case 2}
\end{table}
\subsection{Case 3: Earth to Mars}
\begin{table}[H]
\centering
\begin{tabular}{|lc|}
\hline
Departure date              & 2031 Jan 23                \\ 
Arrival date                & 2031 Aug 01 \\ 
$\Delta$t                    & 190 days                   \\ 
$r_1$                          & (0.5264, 0.8316, 0.0001)  \\ 
$r_2$                          & (0.0108, 1.4542,0.0309)   \\ \hline
\end{tabular}
\caption{Data provided for case 3}
\end{table}

\begin{table}[H]
\centering
\begin{tabular}{|lc|}
\hline
Semimajor axis       &  1.24706 AU
   \\ 
Eccentricity              &   0.381113
    \\ 
$\theta _0$      &  282.584
 \degree      \\
$\omega$            & 77.5612
\degree                            \\ 
Inclination                & 2.29274
\degree                             \\ 
$\Omega$            & 57.8117
\degree                                   \\ 
Heliocentric velocity at departure (km/s) & (-48.5598,30.3938,2.29362) \\ 
Heliocentric velocity at arrival (km/s)&   (20.646,-0.03366,-0.70028) \\
\hline
\end{tabular}
\caption{Results computed for case 3}
\end{table}
\subsection{Case 4: Earth to Mars}

\begin{table}[H]
\centering
\begin{tabular}{|lc|}
\hline
Departure date              & 2025 Jul 18                \\ 
Arrival date                & 2025 Oct 21 \\ 
$\Delta$t                    & 95 days                   \\ 
$r_1$                          & (0.4342, 0.9188, 0.0001)  \\ 
$r_2$                          & (0.6775, 1.3571,0.0118)   \\ \hline
\end{tabular}
\caption{Data provided for case 4}
\end{table}

\begin{table}[H]
\centering
\begin{tabular}{|lc|}
\hline
Semimajor axis       &  122.519 AU   \\ 
Eccentricity              &       -1.0008\\ 
$\theta _0$      &   360.052\degree      \\
$\omega$            & -0.0356\degree                            \\ 
Inclination                & 19.6087\degree                             \\ 
$\Omega$            & 64.7217\degree                                   \\ 
Heliocentric velocity at departure (km/s) & -\\ 
Heliocentric velocity at arrival (km/s)&    -\\
\hline
\end{tabular}
\caption{Results computed for case 4}
\end{table}
Results partially converged. Code, doesn't perform well in this case.
\subsection{Case 5: Earth to Venus}
 \begin{table}[H]
\centering
\begin{tabular}{|lc|}
\hline
Departure date              & 2023 May 27                \\ 
Arrival date                & 2023 Nov 01 \\ 
$\Delta$t                    & 158 days                   \\ 
$r_1$                          & (-0.4255, -0.9194, 0.0000)  \\ 
$r_2$                          & (0.0356, 0.7189,0.0079)   \\ \hline
\end{tabular}
\caption{Data provided for case 5}
\end{table}

\begin{table}[H]
\centering
\begin{tabular}{|lc|}
\hline
Semimajor axis       &   217.426 AU
  \\ 
Eccentricity              &   -0.847689
    \\ 
$\theta _0$      & 360.007
  \degree      \\
$\omega$            & -0.00657141
\degree                            \\ 
Inclination                & 1.67824
\degree                             \\ 
$\Omega$            & 245.165
\degree                                   \\ 
Heliocentric velocity at departure (km/s) & (2.45241,-1.13498,0.0791754) \\ 
Heliocentric velocity at arrival (km/s)&  (4.80202,-5.12288,0.190725) \\
\hline
\end{tabular}
\caption{Results computed for case 5}
\end{table}
\subsection{Case 6: Mars to Earth}

\begin{table}[H]
\centering
\begin{tabular}{|lc|}
\hline
Departure date              & 2033 Jan 18                \\ 
Arrival date                & 2033 Aug 28 \\ 
$\Delta$t                    & 222 days                   \\ 
$r_1$                          & (1.5831, 0.3913, 0.0306)  \\ 
$r_2$                          & (0.9123, 0.4340,0.0000)   \\ \hline
\end{tabular}
\caption{Data provided for case 6}
\end{table}
 
\begin{table}[H]
\centering
\begin{tabular}{|lc|}
\hline
Semimajor axis       &  1.31867 AU
   \\ 
Eccentricity              &     0.236873
  \\ 
$\theta _0$      &   180
\degree      \\
$\omega$            & 371.607
\degree                            \\ 
Inclination                & 5.3505
\degree                             \\ 
$\Omega$            & 205.441
\degree                                   \\ 
Heliocentric velocity at departure (km/s) & (-6.43846,8.513846,-0.97904) \\ 
Heliocentric velocity at arrival (km/s)&   (32.4436,-28.4563,3.71199) \\
\hline
\end{tabular}
\caption{Results computed for case 6}
\end{table}
\subsection{Case 7: Mars to Earth}

\begin{table}[H]
\centering
\begin{tabular}{|lc|}
\hline
Departure date              & 2030 Nov 20                \\ 
Arrival date                & 2031 Jul 06 \\ 
$\Delta$t                    & 228 days                   \\ 
$r_1$                          & (1.4166, 0.8722, 0.0530)  \\ 
$r_2$                          & (0.2345, 0.9893,0.0001)   \\ \hline
\end{tabular}
\caption{Data provided for case 7}
\end{table}

\begin{table}[H]
\centering
\begin{tabular}{|lc|}
\hline
Semimajor axis       &  1.30864 AU
   \\ 
Eccentricity              &   0.271869
    \\ 
$\theta _0$      &  180
 \degree      \\
$\omega$            &405.199
 \degree                            \\ 
Inclination                & 2.57215
\degree                             \\ 
$\Omega$            & 256.79
\degree                                   \\ 
Heliocentric velocity at departure (km/s) &(-9.20594,-5.75652,-0.343519) \\ 
Heliocentric velocity at arrival (km/s)&   (0.111769
,0.4865
,-0.000106258
) \\
\hline
\end{tabular}
\caption{Results computed for case 7}
\end{table}

\subsection{Case 8: Earth to Mars (hyperbolic)}
 \begin{table}[H]
\centering
\begin{tabular}{|lc|}
\hline
Departure date              & 2021 Nov 26                \\ 
Arrival date                & 2022 Feb 19 \\ 
$\Delta$t                    & 85 days                   \\ 
$r_1$                          & (0.4383, 0.8843, 0.0000)  \\ 
$r_2$                          & (-0.2082, -1.4582,-0.0255)   \\ \hline
\end{tabular}
\caption{Data provided for case 8}
\end{table}

\begin{table}[H]
\centering
\begin{tabular}{|lc|}
\hline
Semimajor axis       &  1.34023 AU
   \\ 
Eccentricity              &  1.44255
     \\ 
$\theta _0$      &   288.925
\degree      \\
$\omega$            &251.075
 \degree                            \\ 
Inclination                &3.16587
 \degree                             \\ 
$\Omega$            & 243.635
\degree                                   \\ 
Heliocentric velocity at departure (km/s) &(0.322509
, 0.650683
,0
)\\ 
Heliocentric velocity at arrival (km/s)&   (-1.20802
,-1.22481
,-0.0297817
) \\
\hline
\end{tabular}
\caption{Results computed for case 8}
\end{table}
\subsection{Case 9: Earth to Mars (hyperbolic)}

\begin{table}[H]
\centering
\begin{tabular}{|lc|}
\hline
Departure date              & 2022 Jan 15                \\ 
Arrival date                & 2022 Apr 20 \\ 
$\Delta$t                    & 95 days                   \\ 
$r_1$                          & (-0.4079, 0.8950, 0.0000)  \\ 
$r_2$                          & (0.6393, -1.2542,-0.0420)   \\ \hline
\end{tabular}
\caption{Data provided for case 9}
\end{table}

\begin{table}[H]
\centering
\begin{tabular}{|lc|}
\hline
Semimajor axis       &  5.10169 AU
   \\ 
Eccentricity              &      1.11068
 \\ 
$\theta _0$      &  280.991
 \degree      \\
$\omega$            & 259.009
\degree                            \\ 
Inclination                & 34.288
\degree                             \\ 
$\Omega$            & 294.501
\degree                                   \\ 
Heliocentric velocity at departure (km/s) &(-16.804
, -7.65848
,-12.5916
)\\ 
Heliocentric velocity at arrival (km/s)&   (36.9836
, 14.0711
,26.9253
)\\
\hline
\end{tabular}
\caption{Results computed for case 9}
\end{table}

\section{Complementary calculations}
	\subsection{Earth SOI scape parameters}
	\paragraph{} Applying the equations from the 1st geocentric stage, of the PCA. The first given example, Earth to Mars trip, planetary departure parameters are determined.
	
	\begin{table}[H]
		\centering
		\begin{tabular}{|lc|}
			\hline
			Departure date              & 2020 Jul 19                \\ 
			Arrival date                & 2021 Gen 25 \\ 
			$\Delta$t                    & 190 days                   \\ 
			$r_0$                          & 300 km  \\ 
			$I_{sp}$                          & 300 s  \\
			 \hline
		\end{tabular}
		\caption{Data provided for first example plus assumptions }
	\end{table}
	Note that $r_0$ is the initial circular parking orbit of the spacecraft and $I_{sp}$  is the specific impulse.
	
	\begin{table}[H]
		\centering
		\begin{tabular}{|lc|}
			\hline
			Hyperbolic excess speed, $v_\infty$              & 2.9439 km/s                \\ 
			Spacecraft speed on parking orbit, $v_c$                & 7.7298 km/s\\ 
			$\Delta$v                    & 3.5912 km/s                   \\ 
			Perigee of departure hyperbola                          & 6671 km  \\ 
			$\beta$                          & 29.1525 \textdegree \\
			Spacecraft fuel percent among total mass                          & 70.48 \%  \\
			\hline
		\end{tabular}
		\caption{Calculated parameter for the planetary scape.}
	\end{table}
	
	Spacecraft fuel percent among total mass is calculated as
	\begin{equation}
		\%m_{fuel} = 1-e^{(-\frac{\Delta v}{I_{sp}g_0})}
	\end{equation}
	and for this particular example means that the 70.48\% of the total spacecraft weight will be the fuel that will burn in order to scape the Earth SOI.
	\subsection{Launch windows computation}
	